
%% Copyright 2006-2015 Xavier Danaux (xdanaux@gmail.com).
%
% This work may be distributed and/or modified under the
% conditions of the LaTeX Project Public License version 1.3c,
% available at http://www.latex-project.org/lppl/.

\documentclass[11pt,a4paper,sans]{moderncv}        % possible options include font size ('10pt', '11pt' and '12pt'), paper size ('a4paper', 'letterpaper', 'a5paper', 'legalpaper', 'executivepaper' and 'landscape') and font family ('sans' and 'roman')

% moderncv themes
\moderncvstyle{banking}                             % style options are 'casual' (default), 'classic', 'banking', 'oldstyle' and 'fancy'
\moderncvcolor{blue}                               % color options 'black', 'blue' (default), 'burgundy', 'green', 'grey', 'orange', 'purple' and 'red'

\usepackage{tabto}
%% end of file `template.tex'.
% For header:
\usepackage{fancyhdr}
\usepackage{xcolor}

\pagestyle{fancy}
\fancyhf{}
\renewcommand{\headrulewidth}{0pt}
%\fancyhf{}
\rhead{\color{gray}Klabacka}
\lhead{\color{gray}CV}
\rfoot{Page \thepage}
%\renewcommand{\familydefault}{\sfdefault}         % to set the default font; use '\sfdefault' for the default sans serif font, '\rmdefault' for the default roman one, or any tex font name
%\nopagenumbers{}                                  % uncomment to suppress automatic page numbering for CVs longer than one page

% character encoding
%\usepackage[utf8]{inputenc}                       % if you are not using xelatex ou lualatex, replace by the encoding you are using
%\usepackage{CJKutf8}                              % if you need to use CJK to typeset your resume in Chinese, Japanese or Korean

% adjust the page margins
\usepackage[scale=0.75]{geometry}

%\setlength{\hintscolumnwidth}{3cm}                % if you want to change the width of the column with the dates
%\setlength{\makecvheadnamewidth}{10cm}            % for the 'classic' style, if you want to force the width allocated to your name and avoid line breaks. be careful though, the length is normally calculated to avoid any overlap with your personal info; use this at your own typographical risks...
\name{Randy}{Klabacka}
\title{\Large{curriculum vitae}}                               % optional, remove / comment the line if not wanted
%\phone[mobile]{+1~(234)~567~890}                   % optional, remove / comment the line if not wanted; the optional "type" of the phone can be "mobile" (default), "fixed" or "fax"
\email{randy.klabacka@utahtech.edu}                               % optional, remove / comment the line if not wanted
\homepage{randyklabacka.com}                         % optional, remove / comment the line if not wanted
%\social[twitter]{@HumbleHerper}                             % optional, remove / comment the line if not wanted
\social[github]{rklabacka}                              % optional, remove / comment the line if not wanted
\address{Assistant Professor}{Utah Tech University, St. George, UT}{435-879-4828}% optional, remove / comment the line if not wanted; the "postcode city" and "country" arguments can be omitted or provided empty
%\extrainfo{additional information}                 % optional, remove / comment the line if not wanted
%\photo[64pt][0.4pt]{picture}                       % optional, remove / comment the line if not wanted; '64pt' is the height the picture must be resized to, 0.4pt is the thickness of the frame around it (put it to 0pt for no frame) and 'picture' is the name of the picture file
%\quote{Some quote}                                 % optional, remove / comment the line if not wanted


% bibliography with mutiple entries
% \usepackage{multibib}
% \newcites{book,misc}{{Books},{Others}}
% Import biblatex package.
% -- backend=biber - biber command used to compile bibliography
% -- style=authoryear-comp - enumerating style in bibliography
% -- sorting=ydnt - sort by year
% -- maxbibnames=99 - allow all author names in bibliography
\usepackage[american]{babel}
\usepackage[backend=biber,style=authoryear-comp, sorting=ymdnt, maxbibnames=99]{biblatex}
\DeclareSortingTemplate{ymdnt}{
  \sort{
    \field{presort}
  }
  \sort[final]{
    \field{sortkey}
  }
  \sort[direction=descending]{
    \field{sortyear}
    \field{year}
  }
  \sort[direction=descending]{
    \field{sorttitle}
    \field{title}
  }
  \sort{
    \field[padside=left,padwidth=2,padchar=0]{month}
    \literal{00}
  }
  \sort{
    \field[padside=left,padwidth=2,padchar=0]{day}
    \literal{00}
  }
  \sort{
    \field[padside=left,padwidth=4,padchar=0]{volume}
    \literal{0000}
  }
}

% Include bibliographies (this isn't necessary- bibliographies can be included in separate file
% -- pdflatex will create .bib files for these bibliographies separately
% \begin{filecontents}{pubs.bib}
%   @article{grismer2016two,
%     title={Two new Bent-toed Geckos of the Cyrtodactylus pulchellus complex from Peninsular Malaysia and multiple instances of convergent adaptation to limestone forest ecosystems.},
%     author={Grismer, L Lee and Wood, Jr PL and Anuar, Shahrul and Grismer, Marta S and Quah, ES and Murdoch, Matthew L and Muin, Mohd Abdul and Davis, Hayden R and Aguilar, Cesar and Klabacka, Randy and others},
%     journal={Zootaxa},
%     volume={4105},
%     number={5},
%     pages={401--429},
%     year={2016}
%   }
% \end{filecontents}
% Import bibliography resources
\addbibresource{pubs.bib}
\addbibresource{acks.bib}

%% Function to bold author name
%\newcommand{\makeauthorbold}[1]{%
%  \DeclareNameFormat{author}{%
%    \ifthenelse{\value{listcount}=1}
%    {%
%      {\expandafter\ifstrequal\expandafter{\namepartfamily}{#1}{\mkbibbold{\namepartfamily\addcomma\addspace \namepartgiveni}}{\namepartfamily\addcomma\addspace \namepartgiveni}}
%      %
%    }{\ifnumless{\value{listcount}}{\value{liststop}}
%        {\expandafter\ifstrequal\expandafter{\namepartfamily}{#1}{\mkbibbold{\addcomma\addspace \namepartfamily\addcomma\addspace \namepartgiveni}}{\addcomma\addspace \namepartfamily\addcomma\addspace \namepartgiveni}}
%        {\expandafter\ifstrequal\expandafter{\namepartfamily}{#1}{\mkbibbold{\addcomma\addspace \namepartfamily\addcomma\addspace \namepartgiveni\addcomma\isdot}}{\addcomma\addspace \namepartfamily\addcomma\addspace \namepartgiveni\addcomma\isdot}}%
%      }
%    \ifthenelse{\value{listcount}<\value{liststop}}
%    {\addcomma\space}
%  }
%}
%\makeauthorbold{Klabacka}

%----------------------------------------------------------------------------------
%            content
%----------------------------------------------------------------------------------
\begin{document}
%\begin{CJK*}{UTF8}{gbsn}                          % to typeset your resume in Chinese using CJK
%-----       resume       ---------------------------------------------------------
\makecvtitle

\section{Education}
\cventry{}{Department of Biological Sciences}{Ph.D. in Biological Sciences}{2022}{\textit{Auburn University}}{Advisors: Drs. Tonia Schwartz \& Jamie Oaks}  % arguments 3 to 6 can be left empty
\cventry{}{Department of Biology}{B.S. in Biology}{2016}{\textit{Brigham Young University}}{Advisors: Drs. Jack Sites \& Chad Hancock}  % arguments 3 to 6 can be left empty

\section{Professional Appointments}
\cventry{}{Biological Sciences Department}{Assistant Professor}{2023-present}{\textit{Utah Tech University}}{}


\section{Grants, Fellowships, and Scholarships}
\cvitemwithcomment{2023}{UT Faculty Research Fellowship)}{\$7,000}
\small{\tabto{1.25cm}Bioinformatic examination of genome-wide expression and sequence variation in hybrid asexual lizards} \\
\cvitemwithcomment{2020}{EECG Research Award (American Genetics Association)}{\$8,000}
\small{\tabto{1.25cm}Genomic and bioenergetic costs of asexuality in a vertebrate system (\textit{Aspidoscelis})} \\
\cvitemwithcomment{2017}{CMB Peaks of Excellence Research Fellowship (Auburn University)}{\$4,500}
\small{\tabto{1.25cm}Mitonuclear distancing: The baggage of an asexual reproductive strategy} \\
\cvitemwithcomment{2017}{Meredith Birchfield Endowed Fund for Excellence (Auburn Univ Museum of Natural History)}{\$1500}
\small{\tabto{1.25cm}Examining species boundaries in \textit{Draco maculatus}} \\
\cvitemwithcomment{2016}{Office of Research \& Creative Activities Grant (BYU)}{\$1,500}
\small{\tabto{1.25cm}Phylogeny and species boundaries in spotted flying lizards (\textit{Draco maculatus})} \\
\cvitemwithcomment{2012-15}{Undergraduate Academic Scholarships (BYU)}{\$11,987}

\section{Awards}
\cvitemwithcomment{2019}{1st Place - Henri Seibert Competition Systematics \& Evolution Category (SSAR)}{\$200}
\small{\tabto{1.25cm}Riverine barriers as potential drivers of biodiversification in \textit{Draco maculatus}} \\
\cvitemwithcomment{2019}{Trees in the Desert Workshop (NSF - University of Arizona)}{\$1,000}
\small{\tabto{1.25cm}funded workshop (covering travel, lodging, food, and workshop} \\
\cvitemwithcomment{2019}{COSAM Travel Award (Auburn University)}{\$300}
\small{\tabto{1.25cm}Funding to present research at 9th World Congress of Herpetology} \\
\cvitemwithcomment{2017}{NSF Travel Grant (Society of Systematics Biology Meeting)}{\$500}
\small{\tabto{1.25cm}Funding to present research at 2017 SSB meeting} \\
\cvitemwithcomment{2015}{3rd Place - HBLL College of Life Sciences Poster Competition (BYU)}{\$300}
\small{\tabto{1.25cm}Phylogeny and species boundaries in spotted flying lizards (\textit{Draco maculatus})} \\
\cvitemwithcomment{2015}{College of Life Sciences Dean's List (BYU)}{}
\cvitemwithcomment{2014}{REU Supplement Recipient (BYU)}{\$3,000}
\small{\tabto{1.25cm}Phylogeny and biogeography of New World leaf-toed geckos (\textit{Phyllodactylus})} \\

% Publications section using biblatex
\section{Peer-reviewed Publications}
\begin{refsection}[pubs.bib]
\nocite{*}
\renewbibmacro*{in:}{%
   {}
  }
\printbibliography[heading=none]
\end{refsection}

\section{Manuscripts in-review and in-prep}
\begin{refsection}[prep.bib]
\nocite{*}
\renewbibmacro*{in:}{%
   {}
  }
\printbibliography[heading=none]
\end{refsection}

\section{Invited Seminars}
\small{\cvitemwithcomment{2021}{Workshop on Fostering Idealogical Awareness}{Auburn University}}
\footnotesize{\tabto{1.25cm}Teaching evolution to students of faith: How instructors can help students overcome barriers} \\
\small{\cvitemwithcomment{2019}{Museum of Natural Science Seminar Series}{Louisiana State University}}
\footnotesize{\tabto{1.25cm}Riverine barriers as drivers of biodiversification in terrestrial fauna of Southeast Asia}
\small{\tabto{1.25cm}}

% Presentations section using biblatex
\section{Presentations}
\begin{refsection}[pres.bib]
\nocite{*}
\printbibliography[heading=none]
\end{refsection}

\section{Mentorship}
I have mentored 20 undergraduate students in bioinformatics, field biology, and molecular lab work. Three of these undergraduates published research with me as co-authors. Current positions of these students include veterinary school, hydrology technician, M.S. evol/ecol graduate school, undergraduate research assistant, and working on manuscripts for peer-reviewed publications.

\section{Teaching Experience}
\subsection{{Course Instructor}\footnotesize{........................................................................................................................................}}
\begin{itemize}
\item{\cvitemwithcomment{Fall 2023}{BIOL 4320: Scripting for Biologists}{In-person}}
\item{\cvitemwithcomment{Fall 2023}{BIOL 3030: Principles of Genetics}{In-person}}
\item{\cvitemwithcomment{Fall 2023}{BIOL 2300: Fundamentals of Bioinformatics}{In-person}}
\item{\cvitemwithcomment{Summer 2023}{BIOL 3030: Principles of Genetics}{Online}}
\item{\cvitemwithcomment{Spring 2023}{BIOL 4310: Advanced Bioinformatics}{In-person}}
\item{\cvitemwithcomment{Spring 2023}{BIOL 3030: Principles of Genetics}{Blended}}
\item{\cvitemwithcomment{Spring 2022}{BIOL 7180: Scripting for Biologists}{Online and Synchronous}}
\item{\cvitemwithcomment{Fall 2021}{BIOL 3000: Genetics}{Online and Asynchronous}}
\end{itemize}
\subsection{{Teaching Assistantships}\footnotesize{...............................................................................................................Instructor(s)}}
\begin{itemize}
    \item{\cvitemwithcomment{2021}{BIOL 7180: Scripting for Biologists}{Jamie Oaks}}
    \item{\cvitemwithcomment{2020}{BIOL 4020: Vertebrate Biodiversity Lab}{Dan Warner}}
    \item{\cvitemwithcomment{2020}{BIOL 5740/6740: Herpetology Lab}{Jamie Oaks \& Dan Warner}}
    \item{\cvitemwithcomment{2019}{BIOL 4020: Vertebrate Biodiversity Lab}{Joshua Hall}}
    \item{\cvitemwithcomment{2018}{BIOL 5240/6240: Animal Physiology Lab}{Ray Henry}}
    \item{\cvitemwithcomment{2013-2016}{BIO 130 Lab: Principles of Biology}{Keoni Kauwe \& Byron Adams}}
\end{itemize}
\subsection{{Guest Lectures}\footnotesize{...............................................................................................................................Instructor(s)}}
\begin{itemize}
\item{\cvitemwithcomment{2023}{Principles of Biology (BIOL 1610)}{Overview of Bioinformatics}}
\item{\cvitemwithcomment{2023}{Principles of Biology (BIOL 1610)}{Mendel and the Gene Idea}}
\item{\cvitemwithcomment{2023}{Principles of Biology (BIOL 1610)}{Cell Division}}
\item{\cvitemwithcomment{2021}{Mitonuclear Ecology (BIOL 6750)}{The evolution of sex}}
\item{\cvitemwithcomment{2021}{Scripting for Biologists (BIOL 7180)}{Creating python classes \& using random number generators}}
\item{\cvitemwithcomment{2021}{Scripting for Biologists (BIOL 7180)}{Implementing regular expressions}}
\item{\cvitemwithcomment{2021}{Scripting for Biologists (BIOL 7180)}{Introduction to Biopython}}
\item{\cvitemwithcomment{2019}{Vertebrate Biodiversity (BIOL 4020)}{Amphibian Life History Strategies}}
\item{\cvitemwithcomment{2018}{Functional Genomics (BIOL 5850/6850)}{Using high-throughput sequencing for targeted genes}}
\item{\cvitemwithcomment{2018}{Evolution and Systematics (BIOL 3030}{Early evolutionary ideas- Tree thinking}}
\item{\cvitemwithcomment{2016}{Principles of Biology}{The domains of life}}
\item{\cvitemwithcomment{2016}{Principles of Biology}{The central dogma of biology}}
\end{itemize}

% \section{Summary of Teaching Evaluations}
% {\footnotesize{(Complete teaching evaluations are available upon request)}}
% \subsection{Auburn University: Instructor}
% {Students used a Likert scale (1–6; higher number is better) to anonymously respond to the following:\\ (A) "The instructor's overall performance was"\\(B) "My overall learning in the class was"\\(C) "I was prompted to think critically about the course material"\\(D) "I was provided an environment that supported my learning"}\\ \\
% \begin{tabular}{ p{3cm}p{3cm}p{1.5cm}p{1.5cm}p{1.5cm}p{1.5cm}p{1.5cm}  }
%  \hline
% Course & Semester (n) & A & B & C & D \\
%  \hline
%  BIOL 3000 & Fall 2021 (28) & 5.5 & 4.9 & 5.7 & 5.2  \\
%  BIOL 7180 & Spring 2022 (6) & 5.7 & 5.7 & 5.3 & 6.0 \\ 
%  \hline
% \end{tabular} \\
% {\tiny{n = number of students who participated in the survey}} \par
% %  {Select evaluation comments:} 
% %  \begin{itemize}
% %      \item{\footnotesize{\textit{Even as an \textbf{online professor}, he is more personable, helpful, and genuine than many in person professors I've had. The amount of work and effort he puts into his class to give us a fantastic grasp of all the concepts is clear.}}}
% %  \item{\footnotesize{\textit{He was easy to approach and you could tell he really cared about his students. His tests are set up in a manner to \textbf{test your understanding} of a topic rather than to just memorize what he said. I would absolutely take him as a professor again.}}} 
% %  \item{\footnotesize{\textit{Randy provided a great space for us to \textbf{work collaboratively} and understand how to apply the content of the course to our own research. The course was set up in a way that applies to \textbf{real-world use}.}}}
% %  \item{\footnotesize{\textit{He was very helpful whenever an issue arose and was very willing to help. \textbf{He responded to emails very quickly and gave thorough responses} to help whatever the question was. He did a great job and really cared about the success that his students had.}}}
% % \end{itemize}
% \subsection{Auburn University: Teaching Assistant}
% {Students used a Likert scale (1–6; higher number is better) to anonymously respond to the following:\\ (A) "The laboratory instructor was an effective teaching assistant"\\(B) "The laboratory instructor listened and answered student’s questions well"\\(C) "The laboratory instructor enhanced my interest in the subjects covered by labs"\\(D) "The laboratory instructor created an environment that was conducive to learning in the lab" \\ (E) "The laboratory instructor was respectful of students" }\\ \\
% \begin{tabular}{ p{3cm}p{3cm}p{1.5cm}p{1.5cm}p{1.5cm}p{1.5cm}p{1.5cm}  }
%  \hline
% Course & Semester (n) & A & B & C & D & E \\
%  \hline
%  BIOL 7180 & Spring 2021 (9) & 5.8 & 5.8 & 5.7 & 5.8 & 6.0 \\
%  BIOL 4020* & Fall 2020 (21) & 5.7 & 6.0 & 6.0 & 5.7 & 6.0 \\
%  BIOL 5740/6240 & Spring 2019 (20)  & 5.7    & 5.9 & 5.6 & 5.7 & 5.9 \\
%  BIOL 4020 & Fall 2019 (25)  & 5.6    & 5.6 & 5.3 & 5.4 & 5.9  \\
%  \hline
% \end{tabular} \\
% {\tiny{n = number of students who participated in the survey}} \\
%  {\tiny{Teaching evaluations were not made available to TAs for BIOL 5240/6240, BIOL 5600/6600, and BIOL 2501}} \\
%  {\tiny{* During this course I trained a new TA. Evaluation scores are for both of us}} \par
% % {Select evaluation comments:} 
% % \begin{itemize}
% % \item{\footnotesize{\textit{He required a lot out of us for the purpose of helping us \textbf{prepare for the workforce or post-undergraduate education} which I appreciated.}}}
% %\item{\footnotesize{\textit{Randy was very helpful during our time in class and was \textbf{good at communicating} with us when we went online only.}}}
% %\item{\footnotesize{\textit{They were approachable and worked closely with me to make sure I not only had a good understanding of the material but was \textbf{prepared to succeed.}}}}
% %\item{\footnotesize{\textit{I really enjoyed learning from them, and \textbf{found myself digging deeper into subjects}. They encouraged all of us during our coursework and \textbf{provided amazing feedback}. They are clearly \textbf{passionate} about what they do.}}} \\
% %\end{itemize}


% \subsection{Brigham Young University}
% {Students anonymously responded to the question "How well do you feel the lab instructor performed his responsibility overall?" }\\ \\
% \begin{tabular}{ p{3cm}p{3cm}p{2cm}p{2cm}p{2cm}p{2cm}  }
%  \hline
% Course & Semester (n) & Excellent & Pretty Well & Acceptable & Poor \\
%  \hline
%  BIO 130 (Lab) & Fall 2013 (27)  & 93\%    & 7\% & 0 & 0 \\
%  BIO 130 (Lab) & Winter 2014 (14) &   79\%  & 21\%  & 0 & 0 \\
%  BIO 130 (Lecture) & Winter 2015 (21) & 90\% & 5\% & 5\%  & 0 \\
%  BIO 130 (Lecture) & Winter 2016 (23) & 96\% & 4\% & 0 & 0 \\
%  \hline
% \end{tabular} \\
% {\tiny{n = number of students who participated in the survey}} \par
% % {Select evaluation comments:}
% % \begin{itemize}
% % \item{\footnotesize{\textit{Willing to help at all times. Great with answering through emails and \textbf{encouraged and uplifted} other students.}}}
% % \item{\footnotesize{\textit{He went above and beyond to be accessible and help students with their \textbf{individual needs}. He was \textbf{invested in the success} of his students. Randy knew the material well and was extremely \textbf{fair in his grading}.}}}
% % \item{\footnotesize{\textit{I really liked working with Randy cause he has a \textbf{sincere desire to help} and really knows his stuff. I always felt really comfortable approaching him with questions or things I didn't understand.}}}
% % \end{itemize}

\section{Professional Development}
\begin{itemize}
\item{\cvitemwithcomment{2023}{Microcredential Course: Promoting Active Learning}{Organizer: ACUE}}
\begin{itemize}
\item{\footnotesize{Completed course in effective teaching practices focused on techniques in active learning. Satisfied expectations of the course, including successful completion of 6 modules of learning and implementation.}}
\end{itemize}
\item{\cvitemwithcomment{2021}{Fostering ideological awareness - professional workshop}{Organizer: Dr. Abby Beatty}}
\begin{itemize}
\item{\footnotesize{Week-long, inter-institutional workshop where collaborators presented research and collaboratively created open-source course modules for contextualizing societal and ethical impacts of applied biology.}}
\end{itemize}
\item{\cvitemwithcomment{2020}{Inroduction to Discipline-Based Education Research - graduate course}{Instructor: Dr. Cissy Ballen}}
\begin{itemize}
\item{\footnotesize{Semester-long graduate course focused on topics, literature, and methods of discipline-based education research, with an emphasis on active-learning teaching strategies. As part of this course, we published a manuscript on barriers to introductory biology students (see Tracey et al. in Manuscripts In-review section)}}
\end{itemize}
\item{\cvitemwithcomment{2018}{Engaged and Active Student Learning - professional workshop}{Host: AU Biggio Center}}
\begin{itemize}
\item{\footnotesize{Half-day workshop , literature, and methods of discipline-based education research, with an emphasis on active-learning teaching strategies.}}
\end{itemize}
\end{itemize}


\section{Research Assistantships}
\subsection{{Research Focus}\footnotesize{............................................................................................................Principle Investigator(s)}}
\begin{itemize}
    \item{\cvitemwithcomment{Summer 2022}{Museum Curatorial Assistant}{Jon Armbruster and David Laurencio}}
    \item{\cvitemwithcomment{Summer 2021}{Phylogenetics and Functional Genomics}{Jamie Oaks and Tonia Schwartz}}
    \item{\cvitemwithcomment{Summer 2020}{Phylogenetics}{Jamie Oaks}}
    \item{\cvitemwithcomment{Summer 2019}{Phylogenetics}{Jamie Oaks}}
    \item{\cvitemwithcomment{Summer 2018}{Functional Genomics}{Tonia Schwartz}}
    \item{\cvitemwithcomment{2013-16}{Phylogenetic Systematics}{Jack Sites}}
    \item{\cvitemwithcomment{2013-16}{Metabolic Physiology and Bioenergetics}{Chad Hancock}}
\end{itemize}

\section{Field Experience}
\cvitemwithcomment{2021}{Assisted with animal capture and respirometry of \textit{Thamnophis elegans} in CA}{10 days}
\cvitemwithcomment{2021}{Assisted with animal capture processing of 8 \textit{Anolis} species in FL}{5 days}
\cvitemwithcomment{2020}{With team of 3 collected 200 live \textit{Anolis sagrei} for lab breeding colony}{2 days}
\cvitemwithcomment{2019}{Led team of five in NM and TX and collected 50 live \textit{Aspidoscelis} of five species}{1 month}
\cvitemwithcomment{2018}{Led team of four in NM and TX and collected 210 \textit{Aspidoscelis} of 12 species}{2 months}
\cvitemwithcomment{2017}{Led team of two to validate potential \textit{Aspidoscelis} collection localities}{3 weeks}
\cvitemwithcomment{2016}{Collected various herpetofauna for BYU Bean LS Museum in Thailand and Malaysia}{3 weeks}
\cvitemwithcomment{2015}{Collected morphological data from live \textit{Crotalus oreganus lutosus}}{1 day}
\cvitemwithcomment{2014}{Participated in neotropical biology and geology field course in Costa Rica}{2 weeks}
\cvitemwithcomment{2013}{Counted egg masses \& recorded localities for \textit{Rana luteiventris} habitat restoration}{1 day}

\section{Relevant Research Skills}
\subsection{Computational}
\begin{itemize}
	\item{Develop genomic pipelines for read cleaning, assembly, mapping, and variant calling}
	\item{Implement computational tools for functional genomics (e.g., gene expression), population genetics, and phylogenetics with genomic datasets}
	\item{Run scripts on high-performance clusters using slurm and pbs}
	%\item{22 graded credit hours of Computer Science, Bioinformatics, and Computational Statistics}
	\item{Languages:  Python, C++, Bash, R, LaTeX, HTML, git}
\end{itemize}
\subsection{Molecular}
\begin{itemize}
	\item{Perform DNA sequencing techniques (extraction, optimizing quality/quantity for genomic sequencing, PCR, PCR cleanup, big-dye sequencing, will be performing RNA-seq in 2022)}
	\item{Perform mitochondrial isolation, tissue homogenization (for physiology), mitochondrial respirometry, enzyme activity assays, protein assays, and reactive oxygen species assays.}
\end{itemize}
\subsection{Organismal and Museum Collection}
\begin{itemize}
	\item{Capture and formalin fix herpetofauna and maintain ethanol-preserved collection (curate teaching collection while teaching Vertebrate Biodiversity and Herpetology, which contains over 1000 ethanol-preserved fish, amphibians, and reptiles)}
	\item{Isolate blood from lizards (using post-orbital cavity) and perform general animal necropsy and dissection, flash-preserving tissues in liquid nitrogen.}
\end{itemize}
\subsection{Field and Additional}
\begin{itemize}
	\item{Fluently speak Spanish}
	\item{Established inter-institutional field research in TX and NM}
	\item{Led multiple collection- and research-based field trips in TX, NM, and AZ}
\end{itemize}

\section{Outreach and Community Service} 
\cvitemwithcomment{2023}{Bioinformatics guest lecture}{UT STEM Outreach; Hurricane HS Comp Sci class}
\cvitemwithcomment{2023}{Bioinformatics Q\&A}{Zion International Program; Japanese visiting students}
\cvitemwithcomment{2023}{Museum tour guide}{Museum of Natural Sciences; Utah Tech University}
\cvitemwithcomment{2023}{Virgin river litter collector}{Bi-annual Virgin River Cleanup; UT Biological Sciences Department}
\cvitemwithcomment{2020-present}{QuickGRITS podcast: \href{http://randyklabacka.com/quickGRITS}{\color{blue}link} (available on Spotify)} {Creator}
\cvitemwithcomment{2022}{"Gross Out Camp" Museum Instructor}{Fresh Air family; Auburn, AL}
\cvitemwithcomment{2020}{Chief Science Officers "Zoom In On Science" Guest}{SciTech Institute; Kenya}
\cvitemwithcomment{2020}{Chief Science Officers "Zoom In On Science" Guest}{SciTech Institute; Sonora, Mexico}
\cvitemwithcomment{2019}{Volunteer Field Ornithology TA}{UTEP-IMRS Field Biology Course}
\cvitemwithcomment{2018}{Volunteer Field Herpetology TA}{UTEP-IMRS Field Biology Course}
\cvitemwithcomment{2016}{Reptile and Amphibian Studies Scout Merit Badge Instructor}{Boy Scouts of America}

\section{Department and University Service Positions}
\cvitemwithcomment{2023-present}{Bioinformatics Program Coordinator}{Utah Tech University}
\cvitemwithcomment{2023-present}{IACUC Member}{Utah Tech University}
\cvitemwithcomment{2020-2022}{DBS Seminar Committee Grad Representative}{Auburn University}
\cvitemwithcomment{2018-2021}{Member of the Snake Response Team}{Auburn University}
\cvitemwithcomment{2015-16}{Co-president/founder of Life Sciences Pre-Graduate Student Club}{BYU}

\section{Department and University Service Activities}
\cvitemwithcomment{2023}{Lead Organizer and Host for Dr. Perry Ridge visit}{Utah Tech University Forum}
\cvitemwithcomment{2023}{Organizer and Panel Member for Grad School Q\&A}{Utah Tech University}
\cvitemwithcomment{2023}{Panel Member for Undergraduate Research}{Trailblazer Connections; UT Welcome Week}
\cvitemwithcomment{2023}{Biological Sciences Poster Judge}{Biannual Poster Competition; UT Biological Sciences Dept}
\cvitemwithcomment{2023}{STEM Poster Judge}{Trailblazer Symposium; Utah Tech University}
\cvitemwithcomment{2022}{DBS Seminar Host Chair - Brandon Ogbunu Visit (Princeton)}{Auburn University}
\cvitemwithcomment{2022}{Natural History Museum Open House Representative}{Auburn University}
\cvitemwithcomment{2021}{Safe techniques for handling snakes: Instructor}{E. W. Shell Fisheries, Auburn University}
\cvitemwithcomment{2021}{DBS Seminar Host Committee Member - Rebecca Tarvin Visit (UC- Berkeley)}{Auburn University}
\cvitemwithcomment{2019}{Grad Representative - Global Change Biology Hiring Committee}{Auburn University}
\cvitemwithcomment{2019}{STEM Discovery Day instructor}{Auburn University}
\cvitemwithcomment{2018}{DBS Seminar Host Chair - Matt Fujita Visit (UT- Arlington)}{Auburn University}
\cvitemwithcomment{2018}{Natural History Museum Open House Representative}{Auburn University}
\cvitemwithcomment{2018}{DBS Seminar Host Chair - Marjorie Oleksiak Visit (U Miami)}{Auburn University}
\cvitemwithcomment{2017}{DBS Seminar Host Committee Member - Peter Andolfatto Visit (Princeton)}{Auburn University}
\cvitemwithcomment{2017}{Natural History Museum Open House Representative}{Auburn University}
\cvitemwithcomment{2017}{DBS Seminar Host Committee Member - Armin Moczek Visit (Indiana Univ)}{Auburn University}
\cvitemwithcomment{2016}{Natural History Museum Open House Representative}{Auburn University}
\cvitemwithcomment{2015}{Host for the BYU-sponsored "Night at the Museum"}{Monte L. Bean Life Science Museum}
\cvitemwithcomment{2014}{Tour guide for LSB opening- President's Leadership Council dinner}{Brigham Young University}

\section{Professional Memberships}
Society for the Study of Amphibians and Reptiles (SSAR) \\
Society for Integrative and Comparative Biology (SICB) \\
Society of Systematic Biologists (SSB) \\
Society for the Study of Evolution (SSE) \\
American Genetics Association (AGA) 

%\section{Acknowledged In}
%\begin{refsection}[acks.bib]
%\nocite{*}
%\printbibliography[heading=none]
%\end{refsection}

\section{Scholarly Reviews}
Molecular Ecology \\
Biological Journal of the Linnean Society \\
Herpetologica \\
Entomology, Ornithology, \& Herpetology: Current Research \\

\end{document}


